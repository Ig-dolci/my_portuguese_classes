\documentclass{beamer}
\usepackage[utf8]{inputenc}
\usepackage[T1]{fontenc}
\usepackage[brazil]{babel}
\usepackage{graphicx}
\usepackage{amsmath}
\usepackage{listings}
\usepackage{xcolor}

% Configuração do tema
\usetheme{Madrid}
\usecolortheme{default}

% Configuração para código Python
\definecolor{codegreen}{rgb}{0,0.6,0}
\definecolor{codegray}{rgb}{0.5,0.5,0.5}
\definecolor{codepurple}{rgb}{0.58,0,0.82}
\definecolor{backcolour}{rgb}{0.95,0.95,0.92}

\lstdefinestyle{mystyle}{
    backgroundcolor=\color{backcolour},   
    commentstyle=\color{codegreen},
    keywordstyle=\color{magenta},
    numberstyle=\tiny\color{codegray},
    stringstyle=\color{codepurple},
    basicstyle=\ttfamily\footnotesize,
    breakatwhitespace=false,         
    breaklines=true,                 
    captionpos=b,                    
    keepspaces=true,                 
    numbers=left,                    
    numbersep=5pt,                  
    showspaces=false,                
    showstringspaces=false,
    showtabs=false,                  
    tabsize=2
}
\lstset{style=mystyle}

\title{Função Exponencial e Aplicações na Engenharia}
\subtitle{Aula Adaptada (30 min)}
\author{Daiane I. Dolci}
\date{\today}

\begin{document}

\begin{frame}
    \titlepage
\end{frame}

\begin{frame}{Objetivos da Aula (30 min)}
    \begin{enumerate}
        \item \textbf{Conceitos Fundamentais (10 min):} 
        \begin{itemize}
            \item Definição e comportamento gráfico ($a^x$).
            \item O número de Euler ($e$) e a exponencial natural.
        \end{itemize}
        \item \textbf{Modelagem e Aplicação (20 min):} 
        \begin{itemize}
            \item A forma $f(x) = c \cdot e^{kx} + b$.
            \item Interpretação física dos parâmetros.
            \item Estudo de caso: Lei de Resfriamento de Newton.
        \end{itemize}
        \item \textbf{Prática:} Exercícios propostos para fixação extraclasse.
    \end{enumerate}
\end{frame}

\section{1. Definição da Função Exponencial}

\begin{frame}{1. Definição da Função Exponencial}
    \textbf{Definição Formal:}
    Seja $a \in \mathbb{R}$ tal que $a > 0$ e $a \neq 1$. A função exponencial é definida por:
    
    $$f: \mathbb{R} \to (0, +\infty)$$
    $$f(x) = a^x$$
    
    \textbf{Elementos:}
    \begin{itemize}
        \item \textbf{Domínio:} $\mathbb{R}$ (existe para todo $x$).
        \item \textbf{Imagem:} $(0, +\infty)$ (sempre positiva).
        \item \textbf{Base $a$:} Define o crescimento/decaimento.
    \end{itemize}
\end{frame}

\begin{frame}{Comportamento Gráfico}
    \begin{itemize}
        \item \textbf{Crescente ($a > 1$):} Aumenta rapidamente.
        \begin{itemize} \item Ex: $2^x, e^x, 10^x$ \end{itemize}
        \item \textbf{Decrescente ($0 < a < 1$):} Diminui em direção a zero.
        \begin{itemize} \item Ex: $(0,5)^x, (1/2)^x, e^{-x}$ \end{itemize}
    \end{itemize}
    
    \vspace{0.5cm}
    \textbf{Restrições:} $a > 0$ (evitar complexos) e $a \neq 1$ (não ser constante).
\end{frame}

\section{2. O Número de Euler}

\begin{frame}{2. O Número de Euler ($e$)}
    Entre todas as bases, uma é fundamental para a Engenharia:
    
    $$e \approx 2,718$$
    
    \begin{block}{Propriedade Única}
        A função $f(x) = e^x$ tem sua \textbf{taxa de variação igual ao seu próprio valor}.
    \end{block}
    
    Isso a torna ideal para modelar fenômenos naturais de mudança contínua.
\end{frame}

\begin{frame}{Conceito: Assíntota Horizontal}
    \textbf{Definição Intuitiva:}
    Uma linha imaginária que funciona como um "limite" ou "barreira" para o gráfico.
    
    \vspace{0.5cm}
    
    \begin{itemize}
        \item \textbf{Visualmente:} O gráfico se aproxima infinitamente da reta sem tocá-la.
        \item \textbf{Na Engenharia:} Representa o \textbf{Estado de Equilíbrio}.
    \end{itemize}
    
    \begin{exampleblock}{Exemplo Físico}
        No resfriamento de um café, a temperatura cai mas nunca fica menor que a temperatura da sala. A temperatura da sala é a \textbf{assíntota}.
    \end{exampleblock}
\end{frame}

\section{3. Motivação: O Resfriamento do Café}

\begin{frame}{3. Motivação: O Resfriamento do Café}
    Antes da teoria geral, vamos analisar um problema prático.
    
    \textbf{O Problema:}
    Uma xícara de café a \textbf{85$^\circ$C} é colocada em uma sala a \textbf{22$^\circ$C}.
    
    \vspace{0.3cm}
    
    A temperatura $T$ em função do tempo $t$ (min) segue a equação:
    
    $$T(t) = 22 + 63 \cdot e^{-0.05t}$$
    
    \vspace{0.3cm}
    
    \textbf{Observações:}
    \begin{itemize}
        \item O café esfria até atingir a temperatura da sala (22$^\circ$C).
        \item A queda é rápida no início e lenta no final.
    \end{itemize}
\end{frame}

\section{4. A Forma Geral e Modelagem}

\begin{frame}{4. A Forma Geral e Modelagem}
    O exemplo do café ilustra um padrão universal. Podemos generalizar para a \textbf{Forma Geral}:
    
    \begin{block}{Equação Geral}
        $$f(x) = c \cdot a^x + b$$
    \end{block}
    
    \textbf{Significado dos Parâmetros:}
    \begin{enumerate}
        \item \textbf{$b$ (Assíntota):} Valor de equilíbrio ($t \to \infty$).
        \begin{itemize} \item Café: $b=22$ (Temp. da sala). \end{itemize}
        \item \textbf{$c$ (Amplitude):} Diferença inicial ($f(0) - b$).
        \begin{itemize} \item Café: $c = 85 - 22 = 63$. \end{itemize}
        \item \textbf{$a$ (Fator de Crescimento/Decaimento):}
        \begin{itemize} 
            \item Café: $a = e^{-0.05} \approx 0.95$ (Decaimento, pois $a < 1$).
            \item Nota: Na engenharia, frequentemente usamos $a = e^k$.
        \end{itemize}
    \end{enumerate}
\end{frame}

\section{5. Exercícios Propostos}

\begin{frame}{5. Exercícios Propostos (Para Casa)}
    \textbf{Parte 1: Conceitos}
    \begin{itemize}
        \item Propriedades de potência ($e^a \cdot e^b$, etc).
        \item Identificar funções crescentes/decrescentes.
    \end{itemize}
    
    \vspace{0.5cm}
    
    \textbf{Parte 2: Aplicações}
    \begin{itemize}
        \item \textbf{Carga de Capacitor:} $V_C(t) = 9(1 - e^{-2.5t})$
        \item \textbf{Decaimento Radioativo:} $M(t) = 50e^{-0.1t}$
    \end{itemize}
    
    \vspace{0.3cm}
    \centering \small{Resolva os exercícios detalhados no final do Notebook.}
\end{frame}

\section{Extra}

\begin{frame}{6. Material Extra (Opcional): Machine Learning}
    \textbf{Função Sigmoide (Logística)}
    
    $$\sigma(x) = \frac{1}{1 + e^{-x}}$$
    
    \begin{itemize}
        \item Converte valores reais em probabilidades (0 a 1).
        \item Fundamental para Redes Neurais e Classificação.
    \end{itemize}
    
    \vspace{0.5cm}
    % \centering \includegraphics[width=0.6\textwidth]{example-image-a} 
    % Nota: No notebook geramos o gráfico real.
\end{frame}

\begin{frame}
    \centering
    \Huge \textbf{Obrigada!}
    
    \vspace{1cm}
    \normalsize
    Dúvidas?
\end{frame}

\end{document}


O IMT atua nas três áreas do ensino superior, Ensino, Pesquisa e Extensão. Isso se realiza nos cursos de graduação e pós-graduação e ainda no Centro de Pesquisas que, por meio da Divisão de Inovação e Qualidade (DIQ), responsável pelo gerenciamento do relacionamento institucional com empresas, instituições e entidades de classe, tem o objetivo de estimular ações acadêmicas e empresariais por meio de convênios, além de auxiliar os pesquisadores do IMT na submissão de projetos de P&D.

Por tudo isso, gostaríamos de suscintamente conhecer sua potencialidade para contribuir com o IMT de modo mais amplo, seja atuando na pós-graduação, em projetos de pesquisa, serviços à comunidade/empresas, etc. Justifique essa possibilidade de contribuição indicando suas competências e sua pretensão de crescimento na área indicada.

Para conhecer um pouco de cada área do IMT acesse:

Graduação - https://www.maua.br/graduacao
Pós-graduação - https://www.maua.br/pos-graduacao
Centro de pesquisa do IMT - https://www.maua.br/solucoes
Extensão - https://www.maua.br/graduacao/extensao

Colaboração na área de ensino de graduação

Colaboração na área de ensino de pós-graduação

Proposta de Atuação: Ensino de Pós-Graduação e Pesquisa Avançada

Minha formação como Doutora em Ciências pela USP e minha atual posição de Research Associate no Imperial College London me posicionam para contribuir com a pós-graduação do IMT em áreas de fronteira da engenharia computacional. Minha experiência em desenvolvimento de software científico (Firedrake, Pyadjoint) e aplicação em problemas complexos (interação fluido-estrutura, inversão sísmica) permite a oferta de disciplinas e orientações que conectam a teoria matemática rigorosa às demandas industriais de alta tecnologia.

1. Disciplinas Propostas (Eletivas e Especialização)
Com base na minha expertise técnica e publicações em revistas de alto impacto (JFM, IJNME), proponho a criação ou atualização de disciplinas em três eixos principais:

*   Modelagem e Simulação Avançada (CFD & Multifísica):
    *   Foco: Ir além do uso de softwares comerciais ("caixa preta"), ensinando os alunos a implementar e customizar solvers utilizando métodos de elementos finitos modernos.
    *   Ferramentas: Uso do ecossistema Firedrake/PETSc para simulações de alta fidelidade.
*   Otimização e Problemas Inversos:
    *   Foco: Introduzir técnicas de Otimização Baseada em Adjunção (Adjoint-based Optimization) e Assimilação de Dados, cruciais para a indústria 4.0 e setor de energia (como demonstrado em meu trabalho no RCGI com FWI - Full Waveform Inversion).
*   Computação de Alto Desempenho (HPC) para Engenharia:
    *   Foco: Preparar engenheiros para lidar com grandes volumes de dados e simulações massivas, abordando paralelismo (MPI), conteinerização (Docker) e uso de clusters, competências raras e valorizadas no mercado.

2. Orientação e Pesquisa
Minha trajetória de pesquisa internacional permite oferecer aos alunos de pós-graduação:
*   Temas de Tese Relevantes: Orientação em projetos que aplicam métodos numéricos avançados em problemas reais, como energias renováveis, geofísica e bioengenharia.
*   Conexão Internacional: Possibilidade de co-orientações e intercâmbio de conhecimento com grupos de pesquisa de excelência (como o do Imperial College), elevando o nível das publicações do programa.

3. Diferencial para o Programa
Minha atuação visa formar mestres e doutores com perfil "híbrido": engenheiros com profunda base matemática e habilidade de desenvolvimento de software. Esse perfil é essencial para liderar departamentos de P&D em grandes empresas ou seguir carreira acadêmica de excelência.

Colaboração na área de extensão acadêmica (serviços à comunidade/empresas)

Proposta de Atuação: Extensão Tecnológica e Inovação Industrial

A extensão universitária é o canal por onde o conhecimento acadêmico transforma a sociedade. Minha experiência no desenvolvimento de software open source (Firedrake, Pyadjoint) e em projetos de P&D com financiamento industrial (RCGI/Shell) me permite propor ações de extensão que aproximam o IMT das demandas reais do mercado de tecnologia.

1. Software como Serviço à Comunidade (Open Source)
O desenvolvimento e manutenção de bibliotecas científicas públicas é uma forma moderna e escalável de extensão.
*   Proposta: Criar iniciativas onde alunos e professores desenvolvam módulos de simulação open source para resolver problemas de pequenas e médias empresas locais. Isso dá visibilidade global ao IMT (via GitHub/comunidade científica) e oferece soluções acessíveis ao mercado.

2. Cursos de Curta Duração e Capacitação Profissional
Existe uma lacuna de formação em ferramentas computacionais modernas no mercado de engenharia tradicional.
*   Proposta: Ofertar workshops e cursos de verão abertos à comunidade externa focados em upskilling:
    *   "Python para Engenharia e Automação".
    *   "Introdução à Computação em Nuvem e HPC para Indústria".
    *   "Modelagem Matemática Aplicada a Negócios".

3. Consultoria e Projetos de P&D (DIQ)
Minha vivência em centros de pesquisa focados em inovação (RCGI) me capacitou a traduzir "dores" da indústria em problemas matemáticos solucionáveis.
*   Proposta: Atuar junto à Divisão de Inovação e Qualidade (DIQ) na prospecção e execução de projetos de consultoria técnica avançada, especialmente em setores que demandam otimização de processos e análise de fluidos, trazendo receitas e parcerias estratégicas para o Instituto.

Colaboração junto ao Centro de Pesquisa do IMT

Proposta de Atuação: Pesquisa Aplicada e Internacionalização

O Centro de Pesquisas do IMT é o motor de inovação da instituição. Minha experiência como pesquisadora em projetos de grande porte (RCGI/FAPESP/Shell e Imperial College) me qualifica para fortalecer a atuação do centro em três frentes estratégicas:

1. Captação de Recursos e Projetos de P&D
Tenho experiência na execução de projetos financiados pela indústria (setor de energia).
*   Proposta: Atuar ativamente na elaboração de propostas para agências de fomento (FAPESP, CNPq) e parceiros privados, focando em temas de alta relevância como Transição Energética e Otimização de Processos Industriais. Minha expertise em métodos adjuntos (otimização) é diretamente aplicável para reduzir custos operacionais em parceiros industriais.

2. Internacionalização da Pesquisa
A ciência de ponta é global.
*   Proposta: Estabelecer pontes de colaboração entre o IMT e o Imperial College London (e outros centros europeus), facilitando o intercâmbio de pesquisadores e a realização de workshops conjuntos. Isso eleva o fator de impacto das publicações do centro e atrai visibilidade internacional.

3. Desenvolvimento de Tecnologia Proprietária vs. Open Source
*   Proposta: Equilibrar o desenvolvimento de soluções específicas para empresas (via DIQ) com a contribuição para softwares livres consolidados (como o Firedrake). Isso garante que o IMT retenha know-how tecnológico interno ao mesmo tempo que se beneficia da inovação colaborativa global.




















