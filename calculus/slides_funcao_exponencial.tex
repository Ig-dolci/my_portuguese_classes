\documentclass{beamer}
\usepackage[utf8]{inputenc}
\usepackage[T1]{fontenc}
\usepackage[brazil]{babel}
\usepackage{graphicx}
\usepackage{amsmath}
\usepackage{listings}
\usepackage{xcolor}

% Configuração do tema
\usetheme{Madrid}
\usecolortheme{default}

% Configuração para código Python
\definecolor{codegreen}{rgb}{0,0.6,0}
\definecolor{codegray}{rgb}{0.5,0.5,0.5}
\definecolor{codepurple}{rgb}{0.58,0,0.82}
\definecolor{backcolour}{rgb}{0.95,0.95,0.92}

\lstdefinestyle{mystyle}{
    backgroundcolor=\color{backcolour},   
    commentstyle=\color{codegreen},
    keywordstyle=\color{magenta},
    numberstyle=\tiny\color{codegray},
    stringstyle=\color{codepurple},
    basicstyle=\ttfamily\footnotesize,
    breakatwhitespace=false,         
    breaklines=true,                 
    captionpos=b,                    
    keepspaces=true,                 
    numbers=left,                    
    numbersep=5pt,                  
    showspaces=false,                
    showstringspaces=false,
    showtabs=false,                  
    tabsize=2
}
\lstset{style=mystyle}

\title{Função Exponencial e Aplicações na Engenharia}
\subtitle{Aula Adaptada (30 min)}
\author{Daiane I. Dolci}
\date{\today}

\begin{document}

\begin{frame}
    \titlepage
\end{frame}

\begin{frame}{Objetivos da Aula (30 min)}
    \begin{enumerate}
        \item \textbf{Conceitos Fundamentais (10 min):} 
        \begin{itemize}
            \item Definição e comportamento gráfico ($a^x$).
            \item O número de Euler ($e$) e a exponencial natural.
        \end{itemize}
        \item \textbf{Modelagem e Aplicação (20 min):} 
        \begin{itemize}
            \item A forma $f(x) = c \cdot e^{kx} + b$.
            \item Interpretação física dos parâmetros.
            \item Estudo de caso: Lei de Resfriamento de Newton.
        \end{itemize}
        \item \textbf{Prática:} Exercícios propostos para fixação extraclasse.
    \end{enumerate}
\end{frame}

\section{1. Definição da Função Exponencial}

\begin{frame}{1. Definição da Função Exponencial}
    \textbf{Definição Formal:}
    Seja $a \in \mathbb{R}$ tal que $a > 0$ e $a \neq 1$. A função exponencial é definida por:
    
    $$f: \mathbb{R} \to (0, +\infty)$$
    $$f(x) = a^x$$
    
    \textbf{Elementos:}
    \begin{itemize}
        \item \textbf{Domínio:} $\mathbb{R}$ (existe para todo $x$).
        \item \textbf{Imagem:} $(0, +\infty)$ (sempre positiva).
        \item \textbf{Base $a$:} Define o crescimento/decaimento.
    \end{itemize}
\end{frame}

\begin{frame}{Comportamento Gráfico}
    \begin{itemize}
        \item \textbf{Crescente ($a > 1$):} Aumenta rapidamente.
        \begin{itemize} \item Ex: $2^x, e^x, 10^x$ \end{itemize}
        \item \textbf{Decrescente ($0 < a < 1$):} Diminui em direção a zero.
        \begin{itemize} \item Ex: $(0,5)^x, (1/2)^x, e^{-x}$ \end{itemize}
    \end{itemize}
    
    \vspace{0.5cm}
    \textbf{Restrições:} $a > 0$ (evitar complexos) e $a \neq 1$ (não ser constante).
\end{frame}

\section{2. O Número de Euler}

\begin{frame}{2. O Número de Euler ($e$)}
    Entre todas as bases, uma é fundamental para a Engenharia:
    
    $$e \approx 2,718$$
    
    \begin{block}{Propriedade Única}
        A função $f(x) = e^x$ tem sua \textbf{taxa de variação igual ao seu próprio valor}.
    \end{block}
    
    Isso a torna ideal para modelar fenômenos naturais de mudança contínua.
\end{frame}

\begin{frame}{Conceito: Assíntota Horizontal}
    \textbf{Definição Intuitiva:}
    Uma linha imaginária que funciona como um "limite" ou "barreira" para o gráfico.
    
    \vspace{0.5cm}
    
    \begin{itemize}
        \item \textbf{Visualmente:} O gráfico se aproxima infinitamente da reta sem tocá-la.
        \item \textbf{Na Engenharia:} Representa o \textbf{Estado de Equilíbrio}.
    \end{itemize}
    
    \begin{exampleblock}{Exemplo Físico}
        No resfriamento de um café, a temperatura cai mas nunca fica menor que a temperatura da sala. A temperatura da sala é a \textbf{assíntota}.
    \end{exampleblock}
\end{frame}

\section{3. Motivação: O Resfriamento do Café}

\begin{frame}{3. Motivação: O Resfriamento do Café}
    Antes da teoria geral, vamos analisar um problema prático.
    
    \textbf{O Problema:}
    Uma xícara de café a \textbf{85$^\circ$C} é colocada em uma sala a \textbf{22$^\circ$C}.
    
    A temperatura $T$ em função do tempo $t$ segue a \textbf{\href{http://www2.pelotas.ifsul.edu.br/denise/caloretemperatura/resfriamento.pdf}{Lei de Resfriamento de Newton}}:
    
    $$T(t) = T_{amb} + (T_{inicial} - T_{amb}) \cdot e^{-kt}$$
    
    \vspace{0.2cm}
    
    Substituindo os valores ($T_{amb}=22$, $T_{inicial}=85$):
    
    $$T(t) = 22 + (85 - 22) \cdot e^{-0.05t}$$
    $$T(t) = 22 + 63 \cdot e^{-0.05t}$$
    
    \textbf{Observações:}
    \begin{itemize}
        \item O café esfria até atingir a temperatura da sala (22$^\circ$C).
        \item A queda é rápida no início e lenta no final.
    \end{itemize}
\end{frame}

\section{4. A Forma Geral e Modelagem}

\begin{frame}{4. A Forma Geral e Modelagem}
    O exemplo do café ilustra um padrão universal. Podemos generalizar para a \textbf{Forma Geral}:
    
    \begin{block}{Equação Geral}
        $$f(x) = c \cdot a^x + b$$
    \end{block}
    
    \textbf{Significado dos Parâmetros:}
    \begin{enumerate}
        \item \textbf{$b$ (Assíntota):} Valor de equilíbrio ($t \to \infty$).
        \begin{itemize} \item Café: $b=22$ (Temp. da sala). \end{itemize}
        \item \textbf{$c$ (Diferença Inicial / Escala):} $c = f(0) - b$.
        \begin{itemize} \item Café: $c = 85 - 22 = 63$. \end{itemize}
        \item \textbf{$a$ (Fator de Crescimento/Decaimento):}
        \begin{itemize} 
            \item Café: $a = e^{-0.05} \approx 0.95$ (Decaimento, pois $a < 1$).
            \item Nota: Na engenharia, frequentemente usamos $a = e^k$.
        \end{itemize}
    \end{enumerate}
\end{frame}

\section{5. Exercícios Propostos}

\begin{frame}{5. Exercícios Propostos (Para Casa)}
    \textbf{Parte 1: Conceitos}
    \begin{itemize}
        \item Propriedades de potência ($e^a \cdot e^b$, etc).
        \item Identificar funções crescentes/decrescentes.
    \end{itemize}
    
    \vspace{0.5cm}
    
    \textbf{Parte 2: Aplicações}
    \begin{itemize}
        \item \textbf{Carga de Capacitor:} $V_C(t) = 9(1 - e^{-2.5t})$
        \item \textbf{Decaimento Radioativo:} $M(t) = 50e^{-0.1t}$
    \end{itemize}
    
    \vspace{0.3cm}
    \centering \small{Resolva os exercícios detalhados no final do Notebook.}
\end{frame}

\section{Extra}

\begin{frame}{6. Material Extra (Opcional): Machine Learning}
    \textbf{Função Sigmoide (Logística)}
    
    $$\sigma(x) = \frac{1}{1 + e^{-x}}$$
    
    \begin{itemize}
        \item $x$ é um "score" (ex: $x = \text{Horas} - 4$).
        \item Converte valores reais $(-\infty, +\infty)$ em probabilidades $(0, 1)$.
        \item Fundamental para Redes Neurais e Classificação.
    \end{itemize}
    
    \vspace{0.5cm}
    % \centering \includegraphics[width=0.6\textwidth]{example-image-a} 
    % Nota: No notebook geramos o gráfico real.
\end{frame}

\begin{frame}
    \centering
    \Huge \textbf{Obrigada!}
    
    \vspace{1cm}
    \normalsize
    Dúvidas?
\end{frame}

\end{document}


O IMT atua nas três áreas do ensino superior: Ensino, Pesquisa e Extensão. Isso se realiza nos cursos de graduação e pós-graduação e ainda no Centro de Pesquisas que, por meio da Divisão de Inovação e Qualidade (DIQ), responsável pelo gerenciamento do relacionamento institucional com empresas, instituições e entidades de classe, tem o objetivo de estimular ações acadêmicas e empresariais por meio de convênios, além de auxiliar os pesquisadores do IMT na submissão de projetos de P&D.

Por tudo isso, gostaríamos de sucintamente conhecer sua potencialidade para contribuir com o IMT de modo mais amplo, seja atuando na pós-graduação, em projetos de pesquisa, serviços à comunidade/empresas, etc. Justifique essa possibilidade de contribuição indicando suas competências e sua pretensão de crescimento na área indicada.

Para conhecer um pouco de cada área do IMT acesse:

Graduação - https://www.maua.br/graduacao
Pós-graduação - https://www.maua.br/pos-graduacao
Centro de pesquisa do IMT - https://www.maua.br/solucoes
Extensão - https://www.maua.br/graduacao/extensao


Colaboração na área de ensino de graduação

Contexto: Minha formação em Matemática (Licenciatura e Mestrado) e experiência prévia como docente e tutora permitem-me transitar com facilidade entre disciplinas fundamentais (Cálculo, Álgebra Linear) e aplicadas. Posso contribuir para a modernização do currículo de engenharia introduzindo o "pensamento computacional" desde o início.

Proposta: Integrar o uso de linguagens modernas, como Python, e bibliotecas de computação científica (NumPy, SciPy) no ensino de Cálculo e Métodos Numéricos, transformando a teoria abstrata em laboratórios práticos de simulação. Isso prepara o aluno para o mercado atual, onde a modelagem matemática e a programação são indissociáveis.


Colaboração na área de ensino de pós-graduação

Contexto: Minha formação como Doutora em Ciências pela USP e minha atual posição de Research Associate no Imperial College London posicionam-me para contribuir com a pós-graduação do IMT em áreas da engenharia computacional. Minha experiência em desenvolvimento de software científico (Firedrake, Pyadjoint) e aplicação em problemas complexos (interação fluido-estrutura, problemas inversos) permite a oferta de disciplinas e orientações que conectam a teoria matemática às demandas industriais de alta tecnologia.

Proposta: Com base na minha expertise técnica e publicações em revistas de alto impacto, proponho a criação ou atualização de disciplinas em eixos como:

- Otimização e Problemas Inversos: Introduzir técnicas de Otimização Baseada em Diferenciação Automática e Assimilação de Dados.
- Computação de Alto Desempenho (HPC) para Engenharia: Preparar futuros engenheiros para lidar com grandes volumes de dados e simulações massivas, abordando paralelismo (MPI) e uso de clusters, competências valorizadas no mercado.

Além disso, ofereço orientação em projetos de tese que apliquem métodos numéricos avançados em problemas de controle e otimização aplicados à engenharia.


Colaboração na área de extensão acadêmica (serviços à comunidade/empresas)

Contexto: Acredito que o desenvolvimento e a manutenção de bibliotecas científicas públicas e a capacitação profissional são formas modernas e escaláveis de extensão. Existe uma lacuna de formação em ferramentas computacionais modernas no mercado de engenharia tradicional.

Proposta:

- Software como Serviço à Comunidade (Open Source): Criar iniciativas onde alunos e professores desenvolvam módulos de simulação open source para resolver problemas na engenharia. Isso confere visibilidade ao IMT (via GitHub/comunidade científica) e oferece soluções acessíveis.
- Cursos de Curta Duração: Ofertar workshops e cursos de verão abertos à comunidade externa, focados, por exemplo, em "Python para problemas da Engenharia".


Colaboração junto ao Centro de Pesquisa do IMT

Contexto: Minha experiência como pesquisadora em projetos de grande porte (RCGI/FAPESP/Shell e Imperial College) qualifica-me para fortalecer a atuação do centro com pesquisas de ponta.

Proposta: Focar no desenvolvimento de ferramentas computacionais avançadas para a solução de problemas complexos nas engenharias, buscando parcerias com a indústria e fomento em agências de pesquisa.

