\documentclass{beamer}
\usepackage[utf8]{inputenc}
\usepackage[T1]{fontenc}
\usepackage[brazil]{babel}
\usepackage{graphicx}
\usepackage{amsmath}
\usepackage{listings}
\usepackage{xcolor}

% Configuração do tema
\usetheme{Madrid}
\usecolortheme{default}

% Configuração para código Python
\definecolor{codegreen}{rgb}{0,0.6,0}
\definecolor{codegray}{rgb}{0.5,0.5,0.5}
\definecolor{codepurple}{rgb}{0.58,0,0.82}
\definecolor{backcolour}{rgb}{0.95,0.95,0.92}

\lstdefinestyle{mystyle}{
    backgroundcolor=\color{backcolour},   
    commentstyle=\color{codegreen},
    keywordstyle=\color{magenta},
    numberstyle=\tiny\color{codegray},
    stringstyle=\color{codepurple},
    basicstyle=\ttfamily\footnotesize,
    breakatwhitespace=false,         
    breaklines=true,                 
    captionpos=b,                    
    keepspaces=true,                 
    numbers=left,                    
    numbersep=5pt,                  
    showspaces=false,                
    showstringspaces=false,
    showtabs=false,                  
    tabsize=2
}
\lstset{style=mystyle}

\title{Função Exponencial e Aplicações na Engenharia}
\subtitle{Aula Adaptada (30 min)}
\author{Daiane I. Dolci}
\date{\today}

\begin{document}

\begin{frame}
    \titlepage
\end{frame}

\begin{frame}{Objetivos da Aula (30 min)}
    \begin{enumerate}
        \item \textbf{Conceitos Fundamentais (10 min):} 
        \begin{itemize}
            \item Definição e comportamento gráfico ($a^x$).
            \item O número de Euler ($e$) e a exponencial natural.
        \end{itemize}
        \item \textbf{Modelagem e Aplicação (20 min):} 
        \begin{itemize}
            \item A forma $f(x) = c \cdot e^{kx} + b$.
            \item Interpretação física dos parâmetros.
            \item Estudo de caso: Lei de Resfriamento de Newton.
        \end{itemize}
        \item \textbf{Prática:} Exercícios propostos para fixação extraclasse.
    \end{enumerate}
\end{frame}

\section{1. Definição da Função Exponencial}

\begin{frame}{1. Definição da Função Exponencial}
    Uma \textbf{função exponencial} possui a variável no expoente.
    
    \begin{block}{Forma Básica}
        $$f(x) = a^x$$
    \end{block}
    
    onde a base $a$ é um número real positivo e diferente de 1 ($a > 0, a \neq 1$).
    
    \vspace{0.5cm}
    
    \textbf{Comportamento (Crescimento vs Decaimento):}
    \begin{itemize}
        \item \textbf{Crescente ($a > 1$):} Aumenta rapidamente.
        \begin{itemize} \item Ex: $2^x, e^x, 10^x$ \end{itemize}
        \item \textbf{Decrescente ($0 < a < 1$):} Diminui em direção a zero.
        \begin{itemize} \item Ex: $(0,5)^x, (1/2)^x, e^{-x}$ \end{itemize}
    \end{itemize}
\end{frame}

\begin{frame}{Restrições da Base $a$}
    Por que existem restrições?
    \begin{enumerate}
        \item \textbf{$a > 0$:} Bases negativas geram números complexos em raízes pares.
        \item \textbf{$a \neq 1$:} Se $a=1$, $f(x)=1^x=1$ (constante), sem interesse para modelagem de crescimento.
    \end{enumerate}
\end{frame}

\section{2. O Número de Euler}

\begin{frame}{2. O Número de Euler e a Exponencial Natural}
    O caso mais importante na Engenharia é a base \textbf{e}.
    
    $$e \approx 2,718$$
    
    \begin{block}{Função Exponencial Natural}
        $$f(x) = e^x$$
    \end{block}
    
    \textbf{Propriedades:}
    \begin{itemize}
        \item Sempre crescente ($e > 1$).
        \item Passa por $(0, 1)$.
        \item Assíntota no eixo x ($y=0$).
    \end{itemize}
\end{frame}

\section{3. Modelagem Matemática}

\begin{frame}{3. Forma Geral e Modelagem}
    Para modelar fenômenos reais, usamos a forma parametrizada:
    
    \begin{block}{Equação de Modelagem}
        $$f(x) = c \cdot e^{kx} + b$$
    \end{block}
    
    \textbf{Significado dos Parâmetros:}
    \begin{enumerate}
        \item \textbf{$b$ (Assíntota Horizontal):} Valor de equilíbrio ($t \to \infty$). Ex: Temp. ambiente.
        \item \textbf{$k$ (Taxa):} 
        \begin{itemize}
            \item $k > 0$: Crescimento.
            \item $k < 0$: Decaimento.
        \end{itemize}
        \item \textbf{$c$ (Amplitude):} Diferença inicial em relação à assíntota.
    \end{enumerate}
\end{frame}

\section{4. Aplicação Prática}

\begin{frame}{4. Aplicação: Lei de Resfriamento de Newton}
    \textbf{Problema:} Um motor a 120°C é desligado em uma sala a 25°C.
    
    $$T(t) = 25 + 95e^{-0.08t}$$
    
    \textbf{Interpretação:}
    \begin{itemize}
        \item \textbf{$b = 25$:} Temperatura da sala (o motor tende a 25°C).
        \item \textbf{$c = 95$:} Diferença inicial ($120 - 25 = 95$).
        \item \textbf{$k = -0.08$:} Taxa de resfriamento (negativa = decaimento).
    \end{itemize}
    
    \vspace{0.5cm}
    \textit{(Visualização gráfica no Notebook)}
\end{frame}

\section{5. Exercícios Propostos}

\begin{frame}{5. Exercícios Propostos (Para Casa)}
    \textbf{Parte 1: Conceitos}
    \begin{itemize}
        \item Propriedades de potência ($e^a \cdot e^b$, etc).
        \item Identificar funções crescentes/decrescentes.
    \end{itemize}
    
    \vspace{0.5cm}
    
    \textbf{Parte 2: Aplicações}
    \begin{itemize}
        \item \textbf{Carga de Capacitor:} $V_C(t) = 9(1 - e^{-2.5t})$
        \item \textbf{Resfriamento de Café:} $T(t) = 22 + 63e^{-0.05t}$
        \item \textbf{Decaimento Radioativo:} $M(t) = 50e^{-0.1t}$
    \end{itemize}
    
    \vspace{0.3cm}
    \centering \small{Resolva os exercícios detalhados no final do Notebook.}
\end{frame}

\section{Extra}

\begin{frame}{6. Material Extra (Opcional): Machine Learning}
    \textbf{Função Sigmoide (Logística)}
    
    $$\sigma(x) = \frac{1}{1 + e^{-x}}$$
    
    \begin{itemize}
        \item Converte valores reais em probabilidades (0 a 1).
        \item Fundamental para Redes Neurais e Classificação.
    \end{itemize}
    
    \vspace{0.5cm}
    % \centering \includegraphics[width=0.6\textwidth]{example-image-a} 
    % Nota: No notebook geramos o gráfico real.
\end{frame}

\begin{frame}
    \centering
    \Huge \textbf{Obrigada!}
    
    \vspace{1cm}
    \normalsize
    Dúvidas?
\end{frame}

\end{document}
