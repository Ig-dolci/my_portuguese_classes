\documentclass{beamer}
\usepackage[utf8]{inputenc}
\usepackage[T1]{fontenc}
\usepackage[brazil]{babel}
\usepackage{graphicx}
\usepackage{amsmath}
\usepackage{listings}
\usepackage{xcolor}

% Configuração do tema
\usetheme{Madrid}
\usecolortheme{default}

% Configuração para código Python
\definecolor{codegreen}{rgb}{0,0.6,0}
\definecolor{codegray}{rgb}{0.5,0.5,0.5}
\definecolor{codepurple}{rgb}{0.58,0,0.82}
\definecolor{backcolour}{rgb}{0.95,0.95,0.92}

\lstdefinestyle{mystyle}{
    backgroundcolor=\color{backcolour},   
    commentstyle=\color{codegreen},
    keywordstyle=\color{magenta},
    numberstyle=\tiny\color{codegray},
    stringstyle=\color{codepurple},
    basicstyle=\ttfamily\footnotesize,
    breakatwhitespace=false,         
    breaklines=true,                 
    captionpos=b,                    
    keepspaces=true,                 
    numbers=left,                    
    numbersep=5pt,                  
    showspaces=false,                
    showstringspaces=false,
    showtabs=false,                  
    tabsize=2
}
\lstset{style=mystyle}

\title{Função Exponencial e Aplicações na Engenharia}
\author{Daiane I. Dolci}
\date{\today}

\begin{document}

\begin{frame}
    \titlepage
\end{frame}

\begin{frame}{Objetivos da Aula}
    \begin{enumerate}
        \item \textbf{Conhecer o Número de Euler ($e$):} Entender seu valor como constante e sua importância na natureza.
        \item \textbf{Definição e Propriedades:} Compreender a função exponencial básica ($a^x$), a natural ($e^x$) e suas restrições.
        \item \textbf{Forma Geral:} Analisar a equação $f(x) = c \cdot a^x + b$ e interpretar o efeito gráfico de seus parâmetros.
        \item \textbf{Aplicações Práticas:} Resolver problemas de engenharia envolvendo resfriamento, circuitos elétricos e decaimento radioativo.
        \item \textbf{Tópicos Avançados:} Visualizar o uso de exponenciais em Machine Learning (Função Sigmoide).
    \end{enumerate}
\end{frame}

\section{O Número de Euler (e)}

\begin{frame}{1. Preliminar: O Número de Euler ($e$)}
    O número de Euler, representado pela letra \textbf{e}, é uma constante matemática fundamental, assim como o número $\pi$.
    
    \vspace{0.5cm}
    
    Ele é um número irracional (infinitas casas decimais que não se repetem).
    
    \begin{block}{Valor Aproximado}
        $$e \approx 2,718$$
    \end{block}
\end{frame}

\begin{frame}{Por que $e$ é importante na Engenharia?}
    O número \textbf{e} aparece naturalmente em equações que descrevem como as coisas mudam na natureza:
    
    \begin{itemize}
        \item \textbf{Transferência de Calor:} Como um objeto esfria ao longo do tempo.
        \item \textbf{Crescimento e Decaimento:} Populações, radioatividade.
        \item \textbf{Circuitos Elétricos:} Carga e descarga de capacitores.
    \end{itemize}
\end{frame}

\section{Exemplos Aplicados}

\begin{frame}{2. Exemplos Aplicados à Engenharia}
    \textbf{Exemplo 1: Resfriamento de um Motor (Lei de Newton)}
    
    \vspace{0.3cm}
    
    \textbf{Problema:} Um motor a 120°C é desligado em uma sala a 25°C. A temperatura decresce segundo:
    
    \begin{equation*}
        T(t) = 25 + 95e^{-0.08t}
    \end{equation*}
    
    onde $t$ está em minutos.
    
    \vspace{0.5cm}
    
    \textit{(No notebook, visualizamos o gráfico de decaimento desta função)}
\end{frame}

\section{Definição da Função Exponencial}

\begin{frame}{3. Definição da Função Exponencial}
    Uma \textbf{função exponencial} é aquela onde a variável está no expoente.
    
    \begin{block}{Forma Básica}
        $$f(x) = a^x$$
    \end{block}
    
    onde a base $a$ é um número real positivo e diferente de 1 ($a > 0, a \neq 1$).
\end{frame}

\begin{frame}{Restrições para a base $a$}
    \begin{enumerate}
        \item \textbf{Por que $a > 0$?}
        \begin{itemize}
            \item Se a base fosse negativa (ex: $a = -2$), teríamos problemas com expoentes fracionários (raízes pares de números negativos).
            \item Ex: $(-2)^{0,5} = \sqrt{-2} \notin \mathbb{R}$.
        \end{itemize}
        
        \vspace{0.3cm}
        
        \item \textbf{Por que $a \neq 1$?}
        \begin{itemize}
            \item Se $a = 1$, então $f(x) = 1^x = 1$ (Função Constante).
            \item Perde a característica de crescimento/decaimento.
        \end{itemize}
    \end{enumerate}
\end{frame}

\begin{frame}{Exemplo Simples: $f(x) = 2^x$}
    Vamos visualizar uma função exponencial simples onde a base é 2.
    
    \vspace{0.5cm}
    
    \begin{center}
    \begin{tabular}{|c|c|}
        \hline
        $x$ & $f(x) = 2^x$ \\
        \hline
        -3 & $1/8 = 0,125$ \\
        -2 & $1/4 = 0,25$ \\
        -1 & $1/2 = 0,5$ \\
        0 & 1 \\
        1 & 2 \\
        2 & 4 \\
        3 & 8 \\
        \hline
    \end{tabular}
    \end{center}
    
    \vspace{0.2cm}
    \centering \small{Os valores dobram a cada passo inteiro de $x$.}
\end{frame}

\begin{frame}{A Função Exponencial Natural}
    O caso mais importante na engenharia e ciências é quando a base é o número de Euler ($a = e$).
    
    \begin{block}{Função Exponencial Natural}
        $$f(x) = e^x$$
    \end{block}
    
    \textbf{Propriedades:}
    \begin{enumerate}
        \item \textbf{Domínio:} $\mathbb{R}$
        \item \textbf{Imagem:} Reais positivos ($f(x) > 0$)
        \item \textbf{Intercepto y:} Passa por $(0, 1)$
        \item \textbf{Crescimento:} Sempre crescente ($e > 1$)
        \item \textbf{Assíntota:} Eixo x ($y=0$)
        \item \textbf{Derivada:} $\frac{d}{dx}(e^x) = e^x$
    \end{enumerate}
\end{frame}

\section{Forma Geral}

\begin{frame}{4. Forma Geral da Função Exponencial}
    Na modelagem real, usamos parâmetros de escala e deslocamento.
    
    \begin{block}{Equação Geral}
        $$f(x) = c \cdot a^x + b$$
    \end{block}
    
    \textbf{Significado dos Parâmetros:}
    \begin{itemize}
        \item \textbf{$a$ (Base):} Rapidez do crescimento ($a>1$) ou decaimento ($0<a<1$).
        \item \textbf{$c$ (Coeficiente):} Amplitude ou valor inicial. Se negativo, inverte o gráfico.
        \item \textbf{$b$ (Termo Independente):} Deslocamento vertical. Define a \textbf{assíntota horizontal} ($y=b$).
    \end{itemize}
\end{frame}

\section{Exercícios}

\begin{frame}{5. Exercícios - Parte 1: Conceitos Básicos}
    \textbf{Exercício 1: Propriedades de Potência}
    Simplifique:
    \begin{itemize}
        \item a) $e^3 \cdot e^4$
        \item b) $\frac{e^5}{e^2}$
        \item c) $(e^2)^3$
    \end{itemize}
    
    \vspace{0.3cm}
    
    \textbf{Exercício 2: Avaliação de Funções}
    Para $f(x) = 3 \cdot 2^x + 1$, calcule $f(0)$ e $f(1)$.
    
    \vspace{0.3cm}
    
    \textbf{Exercício 3: Crescente ou Decrescente?}
    \begin{itemize}
        \item a) $f(x) = 2^x$
        \item b) $f(x) = (0,5)^x$
        \item c) $f(x) = e^x$
    \end{itemize}
\end{frame}

\begin{frame}{5. Exercícios - Parte 2: Aplicações}
    \textbf{Exercício 4: Carga de Capacitor}
    $$V_C(t) = 9(1 - e^{-2.5t})$$
    \begin{itemize}
        \item Calcule para $t=0$ e $t=1$.
        \item Qual a tensão máxima (assíntota)?
    \end{itemize}
    
    \vspace{0.3cm}
    
    \textbf{Exercício 5: Resfriamento de Café}
    $$T(t) = 22 + 63e^{-0.05t}$$
    \begin{itemize}
        \item Identifique a temperatura ambiente ($b$).
        \item Identifique a temperatura inicial ($t=0$).
    \end{itemize}
\end{frame}

\section{Exemplo Adicional}

\begin{frame}{6. Machine Learning (Função Sigmoide)}
    Usada em classificação binária (probabilidades entre 0 e 1).
    
    \begin{block}{Equação Sigmoide}
        $$\sigma(x) = \frac{1}{1 + e^{-x}}$$
    \end{block}
    
    \textbf{Propriedades:}
    \begin{itemize}
        \item $x \to +\infty \Rightarrow \sigma(x) \to 1$
        \item $x \to -\infty \Rightarrow \sigma(x) \to 0$
        \item $x = 0 \Rightarrow \sigma(0) = 0.5$
    \end{itemize}
\end{frame}

\begin{frame}
    \centering
    \Huge \textbf{Obrigada!}
\end{frame}

\end{document}
